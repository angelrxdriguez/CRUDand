\documentclass[12pt]{article}
\usepackage[utf8]{inputenc}
\usepackage[T1]{fontenc}
\usepackage[spanish]{babel}
\title{Análisis y Espicificación de Requisitos}
\author{Ángel Panadero Rodríguez}
\date{}

\begin{document}
\maketitle

\section*{Análisis de Requisitos}

Gestor de baraja para Android.\\
Ángel Panadero Rodríguez\\
2º Desarrollo de Aplicaciones Multiplataforma

\section*{Descripción del problema}

Mis amigos y yo queremos una forma simple y rápida de gestionar nuestras cartas. Así llevamos la cuenta y registro de las cartas que intercambiamos.\\
Necesitamos una aplicación para intalar en nuestros móviles donde reflejemos nuestras barajas y los intercambios que hacemos entre nosotros.\\
Debe permitir:
\begin{itemize}
\item Dar cartasa de alta en nuestras barajas.
\item Consultar la baraja de otros usuarios.
\item Dar cartas de baja de nuestra baraja (por pérdida o rotura).
\item Poder cambiar parámetros de nuestras cartas (por si nos equivocamos al meterlas).
\item Intercambiar cartas de mi baraja con las de mis amigos.
\end{itemize}

\section*{Requisitos funcionales}

RF-1: Registrar usuarios con su nombre.\\
RF-2: Permitir registrar en la base de datos cartas. Crear cartas.\\
RF-3: Realizar intercambios de cartas de un usuario a otro.\\
RF-4: Visionar las cartas registradas en la base de datos asociadas a un usuario.

\section*{Requisitos no funcionales}

RNF-1: Permitir modificar las cartas creadas en la base de datos.\\
RNF-2: Ver registros de intercambios de cartas entre usuarios.\\
RNF-3: Cada usuario pueda ver de una forma estética las cartas de otros usuarios.

\section*{Casos de Uso}

\begin{tabular}{|p{1cm}|p{4cm}|p{10cm}|}
\hline
\textbf{Nº} & \textbf{Caso de Uso} & \textbf{Descripción} \\
\hline
CU1 & Iniciar sesión & El usuario accede a la aplicación introduciendo su nombre. Si el nombre existe, se cargan sus datos y cartas asociadas. \\
\hline
CU2 & Crear usuario & Permite a un nuevo usuario registrarse introduciendo únicamente su nombre. Se crea su perfil en la base de datos. \\
\hline
CU3 & Añadir carta a la baraja & El usuario puede dar de alta nuevas cartas, introduciendo sus datos (nombre, tipo, poder, rareza, etc.) para añadirlas a su colección. \\
\hline
CU4 & Modificar carta & El usuario puede editar los datos de una carta existente en su baraja (por ejemplo, si cometió un error al registrarla). \\
\hline
CU5 & Eliminar carta & El usuario puede dar de baja una carta por pérdida o rotura, eliminándola de su baraja. \\
\hline
CU6 & Intercambiar cartas con otro usuario & Permite al usuario seleccionar una carta de su baraja y cambiarla por otra carta de un amigo. La aplicación actualiza automáticamente ambas barajas. \\
\hline
\end{tabular}

\section*{Historias de Usuario}

\begin{tabular}{|p{1.2cm}|p{3.8cm}|p{6cm}|p{3cm}|}
\hline
\textbf{ID} & \textbf{Como...} & \textbf{Quiero...} & \textbf{Para...} \\
\hline
HU1 & Usuario & Iniciar sesin con solo mi nombre & Acceder a mi baraja y datos de forma rápida \\
\hline
HU2 & Nuevo & Registrarme con mi nombre & Iniciar la aplicacion  \\
\hline
HU3 & Usuario & Añadir cartas a mi baraja introduciendo sus datos & Mantener actualizada mi colección \\
\hline
HU4 & Usuario & Modificar los datos de una carta existente & Corregir errores al registrarla \\
\hline
HU5 & Usuario & Eliminar una carta de mi baraja  & Mantener un inventario ordenado \\
\hline
HU6 & Usuario & Intercambiar una carta con un amigo & Actualizar  barajas y llevar control de intercambios \\
\hline
\end{tabular}


\end{document}
